\documentclass[oribibl]{llncs}
\usepackage{amsmath}
\usepackage[T1]{fontenc}
\usepackage{makeidx}  % allows for indexgeneration
\usepackage[ruled,vlined]{algorithm2e}
\usepackage[utf8]{inputenc}
\usepackage{pdfpages}
\usepackage{float}
\usepackage{enumitem}
\usepackage{url}
\usepackage[multiple]{footmisc} % DOUBLE FOOTNOTES ACROSS THE SKY

\pagestyle{headings}
\renewcommand{\footnotemargin}{3.99pt}
\renewcommand{\labelitemi}{$\bullet$}

\title{GPU Optimization of Base Form 820 \\* Collective Artificial Spouse Retirement Plan}

\author{Niklas Schalck Johansson, Nikolaj Aaes and Hildur Flemberg\\
\email{\{nsjo, niaa, hufl\}@itu.dk}}
\institute{IT University of Copenhagen}

\setcounter{tocdepth}{3}
\makeatletter
\renewcommand*\l@author[2]{}
\renewcommand*\l@title[2]{}
\makeatletter

\begin{document}
	\maketitle
	\tableofcontents
	
	% noter:
	% Command + >/< skifter mellem PDF og .txt i textmate \\
	% http://www.stdout.org/~winston/latex/latexsheet.pdf er et cheatsheet for \LaTeX{}
	
	\begin{abstract}
		Abstract is the executive summary of the entire paper. As a summary it should encompass not only the main result, but also the problem, and the evaluation. Shaw [4] reports that good ICSE papers discussed evaluation already in their abstract. This is important in software engineering, because of the emphasis put on the evaluation of results.
A good abstract is readable for non-experts, as it increases the chances that someone will build on your work (often application opportunities appear in other areas).
I try to have four sentences in my abstract. The first states the problem. The second states why the problem is a problem. The third is my startling sentence. The fourth states the implication of my startling sentence.

Simon Peyton Jones summarizes this idea in the following points

Statetheproblem
Saywhyit’saninterestingproblem 
Saywhatyoursolutionachieves
Saywhatfollowsfromyoursolution
		\label{abstract}
				\keywords{CUDA, parallelism, threads, blocks, insurance, capital, software, programming languages, Runge Kutta 4th order}
	\end{abstract}
	
	\section{Introduction}
	
	%!TEX root = /Users/Nikolaj/Developer/GPU-Project/Report/Report.tex
This report is written at the IT University of Copenhagen (ITU) in the spring term of 2013 as a CUDA project supervised by Peter Sestoft from ITU and Hans Henrik Brandenborg Sørensen from the Technical University of Denmark (DTU). The report is addressed to people interested in exploiting the benefits of using general purpose graphical processing units (GPGPUs) to optimize feasible computations in general, but specifically to people interested in CUDA. \\

Graphical Processing Units (GPUs) are most commonly used to render graphics in computer games. GPGPUs are ``general purpose'' variants that share the same architecture as GPUs but are optimized for computing purposes out of line with the graphics category. Since this project is not about graphics, we will refer to GPGPUs simply as GPUs, for shortness, throughout the rest of the report.

In this project, we take a problem from the life insurance industry that requires a significant amount of time to solve and show how to speed it up using the GPU. The problem concerns life insurance policies for married people, where one of them receives an amount of money some time after the other person has died. The challenge faced by the insurance company is to estimate its current and future reserves as accurately as possible such that it is always able to satisfy the obligations to the insurance holders. Such an estimate is largely dependent on an estimate of when each insurance holder is going to die which can be anytime between signing of the policy and some, possibly large, number of years ahead in time. Furthermore, the time of death is dependent on several variables like age, gender etc. which has to be taken into account when making the estimate.\\

The insurance company has a mathematical model that unifies all these variables and calculates the reserves needed to insure a single insurance holder, should he die at any time between the time where the computation is done and some number of years ahead. The model describes the problem to an acceptable level of detail and can be solved by a computer but it takes a lot of time to do so, even with modern CPUs. Moreover, insurance companies typically issue several thousands of policies each of which needs to be solved separately. When solving the policies, tabulation can be used with an advantage which will be further discussed in Section \ref{reflection}. Thus, the time it takes to solve the model for a single policy has to be as small as possible in order not to increase the complexity of the overall computation, which, at the end of the day, is what the insurance company is interested in. \\

The goal of this project is to analyze the opportunity to speed up the computation for a single insurance holder through parallelism with CUDA based on a working sequential implementation in C\#. That is, we aim to optimize the time it takes to estimate the size of the reserve needed at any time during the life of an insurance holder such that the insurance company can fulfill its obligations when the person dies. To accomplish this goal, we implemented a version of the C\# solution in C to have a non-object oriented working implementation producing the same results as the C\# code and resembling the end product. From this implementation we were able to produce a fully parallelized implementation that achieves a substantial increase in speed compared to the original one.\\

The reader of the report is assumed to know the basic principles of concurrency and parallelism. No CUDA specific knowledge is required. The rest of the report is outlined as follows. Section \ref{background} provides some background information on the mathematical model from the insurance company and a general explanation of the principle behind the 4th order Runge-Kutta method used to solve it. Section \ref{problemdefinition}, \ref{scope} and \ref{assumptions} describes the problem definition, scope and project related assumptions. Section \ref{themath} describes the mathematical model and the 4th order Runge-Kutta method. Section \ref{implementation} explains the C implementation that we made as an intermediate step towards a fully parallel solution. Section \ref{cuda} gives a general introduction to the architecture behind CUDA and an explanation of CUDA specific terminology. Section \ref{realization} describes the parallelized solution which is evaluated and benchmarked in section \ref{testing} through \ref{bandc}. Section \ref{reflection} discusses the quality of the results. Section \ref{futurework} gives some suggestions for future work. Finally, Section \ref{conclusion} concludes the report.

	\label{introduction}
		
	\section{Background}
		
	%!TEX root = /Users/Nikolaj/Developer/GPU-Project/Report/Report.tex
%- Related work, projects that precedes this project, why do we use the technology described, what is this new technology
%- Nævn at vi har fået udleveret matematik beforehand

The goal of this project is to determine the sum of money (called the reserve) the insurance company needs to possess to be able to pay the insurance holder's spouse in the case of his or her death. \\

The payment to the spouse will be in the form of a life annuity which is disbursed from the time of death of the insurance holder, unless the death occurs before the retirement age ($r$), in which case the money will be disbursed after a grace period ($g$) determined by the insurance company at the time the insurance was taken out. If the insurance holder dies after the retirement age, but within $g$ years, the annuity is disbursed at the time $g + r$. \\

If the insurance holder is not married at the time of death the insurance is forfeited. If the spouse of the insurance holder dies in the grace period, provided one is active, the insurance is forfeited. If the spouse dies before the insurance holder the insurance is forfeited but the insurance holder still has the opportunity to find a new spouse and continue the insurance. \\

In this project we were given a mathematical specification for calculating this reserve \cite{edlu} and a C\# implementation to match the specification. Our task was to convert the C\# implementation to the C programming language and then to the CUDA C extension. The CUDA C implementation should use the GPGPU paralization options to speed up the calculation of the reserver. \\

The result of running the finished solution with the appropriate input parameters will be a series of points that denotes the reserve needed for the insurance company to fulfill all present and future obligations to the insurance holder.

\subsection{Problem definition}
How can CUDA C be utilized to optimize the calculation of the baseform 820 collective artificial spouse retirement plan using the Runge Kutta 4th order method?

\subsection{Scope}
	The scope of this project is to handle a single customer at a time. We do not explore the possibilities for optimizing the process of using the algorithm on several customers at a time. Neither do we explore alternative methods for numerically solving differential equations.
	
\subsection{Assumptions}
This project was developed under several assumptions to limit its size. We assumed that the spouse of an insurance holder was always of the opposite gender. It would now be possible to take same-gender couples into account but it would require som modification to the code. We assumed that the  interest rate is always 5 \%. 



	\label{background}
		
	\section{The Math}
		
	- Description of the insurance math used in the current solution \\
\\
\\The algorithm in this project is used to determine the lump sum of money the insurance company needs to possess to be able to pay the insurance holder's spouse in the case of his or her death. The payment to the spouse will be in the form of a life interest which are disbursed from the time of death of the insurance holder unless the death occurs before the pension age. In that case the money will be disbursed after a grace period determined by the insurance company at the time the insurance was taken out. 
\\
\\ We assume that the spouse of the insurance holder is of the opposite gender. If the insurance holder does not have a spouse at the time of his or her death the insurance is forfeited.
\\
\\The algorithm used here is a 4th order Runge-Kutta solution (indsæt reference) where we use a series of constants determined before any calculation begins:

\begin{center}
\begin{tabular}[t]{|c|c|}
\multicolumn{2}{c}{Constants}\\\hline
\textbf{Name}&\textbf{Meaning}\\\hline
$\tau$&The time of death of the insurance holder\\\hline
$r$&The pension age\\\hline
$g$&The grace period\\\hline
$x$&The age of the insurance holder at calculation time ($t$ = 0)\\\hline
$t$&The time of calculation\\\hline
$h$&The Stepsize of the Runge-Kutta solution\\\hline
\end{tabular}
\end{center}

Apart from this there is also a constant $k$ which are determined by $\tau, r$ and $g$ in the following manner:

\begin{center}
	\begin{tabular}[t]{|c|c|}
		\multicolumn{2}{c}{How $k$ is defined}\\\hline
		\textbf{If this statement holds}&\textbf{then $k$ equals}\\\hline
		$\tau < r$&$g$\\\hline
		$r \leq \tau < r + g$& $r + g - \tau$\\\hline
		$r + g \leq \tau$&0\\\hline
	\end{tabular}
\end{center}
	\label{themath}
	
	BESKRIV HVORDAN EN NORMAL KØRSEL SERUD = EKSEMPEL
	kig evt på en simpson integration når vi kigger på middle da det egentlig ikke er en differential ligning men et integrale
		
	\section{CUDA}
en tråd skal lave et vist stykke arbejde før det kan betale sig at oprette den. Der er overhead for at lave context for hver tråd så hvis den arbejde tråden laver er mindre end at sætte overheaden op, giver det ingen mening.

	%!TEX root = /Users/Nikolaj/Developer/GPU-Project/Report/Report.tex

\subsection{Architecture}
In a traditional computer, CPUs and GPUs coexist as the processing units of the computer and the graphics chip respectively. Historically, the graphics chip has been used to render games and images on screen while the CPU is responsible for a lot of administrative tasks like executing the operating system and the processes running on it. Besides a great deal of floating point operations (flops) this requires that it communicates with other parts of the system through channels that might suffer from high latency which introduces the need for caching. Since caching is a vital function for the CPU it also occupies a lot of chip area, leaving fewer transistors for arithmetic logical units (ALU) to do actual arithmetic operations. \\

Conversely, a large number of ALUs is what makes the GPU effective in doing its job because pixel rendering requires a lot of flops per second to create a smooth graphical user experience. Where the CPU uses large caches to hide latency from other hardware units, where the GPU makes use of warps and small caches to hide latency from read and write operations. The concept of warps will be discussed later in section \ref{sec:kernelandwarps}.

(INSÆT FIGUR fra CUDA BOG)
%The GPU never directly communicates with devices, like the hard drive, but gets its data from on-board memory areas so it does not need a large cache area.\\

Compute Unified Device Architecture (CUDA) is a parallel computing model created by Nvidia in 2006 based on an extension of the C programming language. In a CUDA enabled computing system, the CPU is known as the \emph{host} and GPU(s) are known as \emph{devices}. CUDA programs are executed on the system as whole, meaning that the host and the device cooperate on computing the output. This is advantageous because some computing tasks are only partially suited for parallelization, meaning that such tasks should be split in smaller parts and executed separately on the most favourable piece of hardware for maximum efficiency.\\

The entry point of a program is always a procedure executed by the host. When the execution is started, the host is instructed to delegate suitable parts of the task to one or several devices embedded as code blocks in \emph{kernels} which is the CUDA term for a function executed on a device. If the task computed by the kernel operates on external data, that data must be copied (by the host) into device memory before the kernel is invoked. Likewise, when the kernel terminates, the host is responsible for reading the data back and deallocate the memory areas used.\\

\subsection{Streaming Multiprocessors}
The computing resources of the GPU is divided into a number of streaming multiprocessors (SMP) each responsible for scheduling a number of threads. Typically, the GPU as a whole executes thousands of threads in parallel split between a number of SMPs. Each thread is much more lightweight than normal CPU threads and typically take only a few clock cycles to create and schedule compared to several thousands of cycles on normal CPUs. Thus, thread context switching is a very efficient operation. Other than that, CUDA threads are similar to regular CPU threads with registers etc. All threads execute the same kernel simultaneously but work on different pieces of the data following the ´´Single Program Multiple Data'' (SPMD) paradigm in contrast to CPU threads that are programmed separately. \\

\subsection{Blocks}
Since a kernel is executed by up to millions of threads and each SMP has an upper limit on the number of threads it can execute at a time, SMPs load threads in groups called blocks. Threads within a block can communicate through the shared memory region and will always be executed by the same SMP once assigned. Other examples of properties for thread blocks include synchronization, whereby the threads in the block can be synchronized to wait for each other at a specific place in the kernel code. The capacity of an SMP is limited both in terms of threads and in terms of blocks. E.g if a block contains 128 threads, the SMP has room for 8 blocks, and if the block contains 256 threads, the SMP has room for only 4 blocks.

\subsection{Kernel Invocation and Warps}
\label{sec:kernelandwarps}
When a kernel is invoked, the user specifies two parameters enclosed in angle brakets before the parameter list. The first parameter selects the number of blocks that should be available for thel execution of the kernel. The second parameter selects the number of threads in each block. The configuration of threads and blocks is normally referred to as the kernel grid because each block is identified by three numbers corresponding to coordinates in a 3-dimensional grid. Likewise each thread is identified within a block using 3 thread coordinates. These identifiers are all available in code through predefined variables and are used to distinquish threads from each other when selecting data for the thread to operate on.

When a number of blocks from the grid are loaded by an SMP, the threads of each block are futher split into units of 32 called warps which is the smallest unit of execution in the sense that two threads from the same warp always execute the same instruction. After some time, one or several blocks are put aside and leave room for new blocks to fill the gap. The user is left without tools to choose which blocks should be loaded, in which order and for how long time the blocks should be executed before new blocks take their place. This is all scheduled by the hardware which gives the CUDA programs a strength because programs can be moved between different machines with different GPUs and automatically utilize the hardware available without changing the source code. The more SMPs available on the chip, the more blocks and threads can execute in parallel.

\subsection{Memory}
A CUDA device exposes different kinds of memory with each their latency and access scope. Since latency has a direct influence on performance, considering which types of memory is used in a CUDA program is important to utilize the full performance potential. Besides latency issues, some threads in an SMP might use more memory than is available in which case the number of threads will be automatically reduced, thus affecting the overall performance of the program. 

The memory model in CUDA is shown in Fig. XX. The figure shows a grid of threads divided into 2 blocks with 2 threads each. Memory types include constant memory, global memory, shared memory and thread registers. Global- and constant memory is the only types of memory that can accessed from the host and is the place where the host can drop data to be used in kernel code using functions similar to the well-known malloc function from traditional C. The kernel threads then access the data directly using their block- and thread IDs. Constant memory is fast to read and write to compared to global memory, but is read-only by the device meaning that results from the kernel is written to global memory where it is subsequently picked up by the host.

Global memory suffers from high latency and has a potential for slowing down the kernel execution if not used with care. This is partly made up for by the constant memory which is slightly faster but only read-only by the device. Shared memory and thread registers on the other hand is extremely fast compared to all other memory types. Shared memory is private to each block, limited in size and offers a mean for threads in the block to communicate by writing intermediate results from their computation where other threads can reach it. Thread registers are private to each thread and is used to store frequently accessed variable that does not need to be accessed by other threads. Keywords and functions exist in CUDA C to allocate/deallocate memory and declare a variable as residing in either of the memory types which directly defines the accessibility of the variable to other threads.
	\label{cuda}
		
	\section{Implementation}
		
	- A description of the new implementation and design choices

Husk at nævne at Middle stepsize er hardcoded til 2.
	\label{implementation}
	I den kode der er nu er insurance holder kvinde og spouse mand ALTID
	skriv noget om memory bound og compute bound i rapporten
		
	\section{Testing}
	test på forskellige maskiner
	hvis vi deler steps op i blocks af 32 tråde kan der i worst case scenario være en sidste block med 1 aktiv og 31 inaktive tråde.
	Lav en graf over rest trådene
		
	- A test to ensure that the new implementation produces the same output as the current implementation
	\label{testing}
		
	\section{Benchmarks and Comparison}
		
	- Benchmarking and comparison on speed between the current and new implementation

Tre slags grafer: \\
køretid og stepsize med konstante g,r,x. Tag ca 9 stikprøver. \\
CUDA køretid og rest tråde. \\
køretid og variable g,r,x.
	\label{bandc}
	
	\section{Reflection / Discussion}
	improvements:
	det kan optimeres til at kunne køre flere kunder.
	Simpson løsning i middle?
	
	KUNNE DET VÆRE INTERESSANT AT BRUGE DYNAMIC PROGRAMMING?
	KUNNE MAN GEMME HVILKE KALD TIL INNER DER ER LAVET OG HVIS INNER SÅ KALDES IGEN MED SAMME VÆRDIER, KAN MAN SLIPPE FOR AT REGNE DET UD OG BARE HENTE RESULTATET. DOG SKAL MAN VÆRE OPMÆRKSOM PÅ AT MAN KAN KOMME I EN SITUATION HVOR EN BEREGNING SER AT DER IKKE ER NOGET RESULTAT FOR ET SÆT PARAMETRE OG DERFOR GÅR I GANG MED AT REGNE DET UD, MENS DEN FØRST BEREGNING KØRER KAN ANDRE BEREGNINGER MED SAMME PARAMETRE OG DE NYE BEREGNINGER VIL OGSÅ GERNE GEMME RESULTATET NÅR DE ER FÆRDIGE HVILKET KAN GIVE CONFLICTS (MÅSKE).
	eller
	Det kan være vi skal dokumentere en del om at man kan gemme alle resultater i en tabel og efter et vist antal kunder kan man ende med nok til bare at have opslag i stedet for beregninger. skal denne tabel et ligge på GPU så den kan slås op i konstant tid.
		
		skriv evt noget om autotuning, hvor programmet først tester med forskellige mængder af tråde og blokke og vælger den hurtigste løsning.
		
		Threats to validity???
		kan det betale sig at køre dele af programmet på CPU'en?
		
	%!TEX root = /Users/Nikolaj/Developer/GPU-Project/Report/Report.tex
%kig evt på en simpson integration når vi kigger på middle da det egentlig ikke er en differential ligning men et integrale
%simpson video: http://www.youtube.com/watch?v=ns3k-Lz7qWU

TODO: skriv evt noget om autotuning, hvor programmet først tester med forskellige mængder af tråde og blokke og vælger den hurtigste løsning. \\

%kan det betale sig at køre dele af programmet på CPU'en?

%- Did we achieve what we wanted? what did we discover during the project? What can be changed in future implementations? \\

When computing several customers one after another it could be beneficial to store the result of each calculation of the outer, middle and inner model and what parameters they were given. Whenever the program wanted to calculate value it could check if the calculation was already performed at a previous time and instantly get the result instead of calculating it again. If the result was not found in the storage it could calculate it itself and store the result afterwards for future use. \\ 

This approach is only beneficial when the lookup in the storage is faster than performing the calculation itself. It is also not beneficial if the amount of calculations that are identical, is not large enough. It would be interesting to examine the calculation collision with a large amount of executions with different customers. A drawback with this approach is that the storage can become very large over time and that could potentially lead to slower lookups. Since the approach stops being beneficial when the lookups are slower than the actual calculations it would be smart to examine when this occurs and evaluate what results is stored. If it occurs too fast one might consider to only store the middle or outer results because they are slower to calculate than the inner results. \\

If we examine the differential equation used in the middle model \\$S_\tau^d(t) = g_\tau \int f(\eta|\tau)a_{[\eta] + k}^I(t)d\eta$ it is a constant times an integral. This means that there is no need to use the Runge Kutta method here we could have chosen another approach for integral approximating such as Simpson's rule\cite{simp}. Simpson's rule requires a even number of steps which our middle model will always have as long as the stepsize is hardcoded to 2. Where the Runge Kutta method uses four calculations for each step ($k1$, $k2$, $k3$ and $k4$), Simpson's rule only uses a single calculation. This could potentially make the computing of the middle model four times faster. 

	\label{reflection}
		
	\section{Future work}
	
	%!TEX root = /Users/Nikolaj/Developer/GPU-Project/Report/Report.tex
This project has a limited scope and we would like to document suggestions for future expansions of the project.

\begin{description}
	
\item[Formal verification of correctness of code] \hfill \\
We assume that the code produces the correct results because it is modeled after existing code and because our test produce satisfactory results. Our test give a sense of correctness but without formal verification there is always the possibility that a corner case is not covered and will produce incorrect results. \\

\item[Handling multiple insurance holders] \hfill \\
This implementation is built around the assumption that it would be run for a single insurance holder at a time. If it is to have any real-world use, it must be able to handle multiple insurance holders at the same time. While this is possible with the current implementation is has not been tested or parallelized for multiple insurance holders. It would be interesting to examine how GPUs could be utilized to handle multiple insurance holders in parallel and if the time used could be lower than when handling them sequencially. \\

\item[Variable interest curve] \hfill \\
In this project the interest rate is assumed to be a constant of 5 \%. However in the real world this is not the case. The interest rate is variable over time and the program should be able to take this into account. \\

\item[Switch genders] \hfill \\
When handling insurance holders, being able to switch the gender of the insurance holder and spouse is necessary. This is not a complex extension but it is not a part of the current implementation. The only gender specific code is the \texttt{GmFemale} and \texttt{GmMale} methods which are used to calculate mortality intensities. Changing whether the female or male function should be called for each insurance holder and spouse should be trivial in the implementation. \\

\item[Same-gender marriages] \hfill \\
As an addition to switching gender the implementation would be able to emulate the real world better if it supported same-gender marriages. 
To support same-gender marriages the implementation would have to be changed so that the same method, either \texttt{GmFemale} or \texttt{GmMale}, is called for the insurance holder and spouse. Additionally the method returning the probability that a person is married might be different for same-gender marriages. \\

\item[Upper bound on threads per block] \hfill \\
To avoid unnecessary crashes, the program should contain a mechanism to prevent the user from setting the number of threads per block higher than the number of threads allowed in a single SMP on the active GPU. If it is set higher, the kernel will not run and the implementation should provide an error message.
\end{description}

	\label{futurework}
	
	\section{Conclusion}
	
	In this project, we took a well-defined computing task from the life insurance industry and showed how to speed it up using parallization with CUDA. The problem consisted of estimating the size of the reserves that an insurance company needed to have available at any time during the life of an insurance holder to be able to fulfill its obligations to his or her spouse in case the insurance holder died.\\

The goal of the project was to parallelize the estimation of the reserve for the Base Form 820 using CUDA C in order to decrease the running time significantly. To achieve this goal, we implemented a C version of the provided C\# implementation which we later changed into two CUDA C implementations. We tested the precision and performance of all implementations and found that the OuterPar implementation was by far the fastest and performed 670 times faster than the provided C\# implementation. In terms of precision, OuterPar deviated from the Edlund estimation by approximately 0.05\%.\\

The project was considered a success since the goal of optimizing the running time significantly was reached and the deviation between the result from OuterPar and the result from Edlund is small.
	\label{conclusion}
		
	\section{Glossary}
	
	\begin{itemize}
	\item Forsikringstager - insurance holder
	\item ægtefælle - Spouse
	\item livrente - life interest
	\item dødfaldssum - Death benefit
	\item pause periode - grace period
	\item randbetingelse - boundary condition
\end{itemize}
		
	\begin{thebibliography}{9}
			\bibitem{edlu} EDLUND DOKUMENTET
			\bibitem{pmpp} David B. Kirk, Wen-mei W. Hwu - Programming Massively Parallel Proccesors, A Hands-on Approach - Elsevier Inc. - 2010
	\end{thebibliography}
	
\end{document}