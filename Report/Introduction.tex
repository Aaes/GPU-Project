This report is written at the IT University of Copenhagen (ITU) in the spring term of 2013 in connection with a CUDA project supervised by Peter Sestoft from ITU and Hans Henrik Brandenborg Sørensen from the Technical University of Denmark (DTU). The report is addressed to people interested in exploiting the benefits of using general purpose graphical processing units (GPGPUs)\footnote{Graphical Processing Units (GPUs) are most commonly used to render graphics in computer games. GPGPUs are ``general purpose'' variants that share the same architecture as GPUs but are optimized for computing purposes out of line with the graphics category. Since this project is not about graphics, we will refer to GPGPUs simply as GPUs, for shortness, throughout the rest of the report.} to optimize feasible computations in general, but specifically to people interested in CUDA.\\

In this project, we take a problem from the insurance industry that requires a significant amount of time to solve and show how to speed it up using the GPU. The problem concerns life insurance policies for married people, where one of them receives an amount of money some time after the other person has died. The goal, from the view of the insurance company, is to estimate its holding as accurately as possible to always be able to satisfy the obligation to the policyholder. Such an estimate is largely dependent on a guess of when the policyholder is going to die which can be anytime between signing of the policy and some, possibly large, number of years ahead in time. Furthermore, the time of death is dependent on several variables like age, gender etc. \\

The insurance company has a mathematical model that unifies all these variables and can be solved by a computer but it takes a lot of time to do - even with modern CPUs. Moreover, insurance companies typically issue several thousands of policies, each of which has an impact on the complexity of the computation as well as the size of the overall reserves the company needs to hold back in case a disbursement is claimed. All these facts together make computing power a valuable entity. The goal of this project is to analyze the opportunity to speed up the computation for a single policyholder through parallelism with CUDA, i.e. to optimize the time it takes to compute the size of the reserves needed at any time during the life of a policyholder such that the insurance company can fulfill its obligations when the person dies.\\

The reader of the report is assumed to know the basic principles of concurrency and parallelism. No CUDA specific knowledge is required.