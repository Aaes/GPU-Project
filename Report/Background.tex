%!TEX root = /Users/Nikolaj/Developer/GPU-Project/Report/Report.tex
%- Related work, projects that precedes this project, why do we use the technology described, what is this new technology
%- Nævn at vi har fået udleveret matematik beforehand

The goal of this project is to determine the sum of money (called the reserve) the insurance company needs to possess to be able to pay the insurance holder's spouse in the case of his or her death. \\

The payment to the spouse will be in the form of a life annuity which is disbursed from the time of death of the insurance holder, unless the death occurs before the retirement age ($r$), in which case the money will be disbursed after a grace period ($g$) determined by the insurance company at the time the insurance was taken out. If the insurance holder dies after the retirement age, but within $g$ years, the annuity is disbursed at the time $g + r$. \\

If the insurance holder is not married at the time of death the insurance is forfeited. If the spouse of the insurance holder dies in the grace period, provided one is active, the insurance is forfeited. If the spouse dies before the insurance holder the insurance is forfeited but the insurance holder still has the opportunity to find a new spouse and continue the insurance. \\

In this project we were given a mathematical specification for calculating this reserve \cite{edlu} and a C\# implementation to match the specification. Our task was to convert the C\# implementation to the C programming language and then to the CUDA C extension. The CUDA C implementation should use the GPGPU paralization options to speed up the calculation of the reserver. \\

The result of running the finished solution with the appropriate input parameters will be a series of points that denotes the reserve needed for the insurance company to fulfill all present and future obligations to the insurance holder.


