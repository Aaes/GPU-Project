%!TEX root = /Users/Nikolaj/Developer/GPU-Project/Report/Report.tex
%Husk at nævne at Middle stepsize er hardcoded til 2.
%Husk at skrive noget om first step der i princippet ikke er en del af matematikken \\
%Husk at nævne at vi i Middle har sat k3 = k2;
%Husk at nævne at det antal steps vi tager kan variere i forhold til matematikken fordi vi tager et "Helt" step hver gang vi laver et firstStep hvor det jo i virkeligheden er en brøkdel af et step. Det gør ikke noget for resultatet, men hvis man tæller steps i programmet vil det ikke give det rigtige antal.
%	I den kode der er nu er insurance holder kvinde og spouse mand ALTID

After we established the mathematical base we implemented a C solution. Instead of implementing the solution directly into CUDA C we chose to implement a C solution first to crush any bugs that were purely C specific. This also made it easier to debug since CUDA C program are generally harder to debug (INDSÆT REF?). \\

The first step was to implement the utility methods used for calculating trivial equation and these were mostly taken directly from the original C\# implementation. The methods in question are \texttt{gTau(double tau)}, \texttt{f(double eta, double tau)}, \texttt{k(double tau, double r, double g)}, \texttt{r\_(double t)}, \texttt{GmFema-\\le(double t)} and \texttt{GmMale(double t)}. These all calculate components in each of the different differential equations.

