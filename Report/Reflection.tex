%!TEX root = /Users/Nikolaj/Developer/GPU-Project/Report/Report.tex
%kig evt på en simpson integration når vi kigger på middle da det egentlig ikke er en differential ligning men et integrale
%simpson video: http://www.youtube.com/watch?v=ns3k-Lz7qWU

%skriv evt noget om autotuning, hvor programmet først tester med forskellige mængder af tråde og blokke og vælger den hurtigste løsning.

%Threats to validity???
%kan det betale sig at køre dele af programmet på CPU'en?

%- Did we achieve what we wanted? what did we discover during the project? What can be changed in future implementations? \\

When computing several customers one after another it could be beneficial to store the result of each calculation of the outer, middle and inner model and what parameters they were given. Whenever the program wanted to calculate value it could check if the calculation was already performed at a previous time and instantly get the result instead of calculating it again. If the result was not found in the storage it could calculate it itself and store the result afterwards for future use. \\ 

This approach is only beneficial when the lookup in the storage is faster than performing the calculation itself. It is also not beneficial if the amount of calculations that are identical, is not large enough. It would be interesting to examine the calculation collision with a large amount of executions with different customers. A drawback with this approach is that the storage can become very large over time and that could potentially lead to slower lookups. Since the approach stops being beneficial when the lookups are slower than the actual calculations it would be smart to examine when this occurs and evaluate what results is stored. If it occurs too fast one might consider to only store the middle or outer results because they are slower to calculate than the inner results. \\

If we examine the differential equation used in the middle model \\$S_\tau^d(t) = g_\tau \int f(\eta|\tau)a_{[\eta] + k}^I(t)d\eta$ it is a constant times an integral. This means that there is no need to use the Runge Kutta method here we could have chosen another approach for integral approximating such as Simpson's rule\cite{simp}. Simpson's rule requires a even number of steps which our middle model will always have as long as the stepsize is hardcoded to 2. Where the Runge Kutta method uses four calculations for each step ($k1$, $k2$, $k3$ and $k4$), Simpson's rule only uses a single calculation. This could potentially make the computing of the middle model four times faster. 
