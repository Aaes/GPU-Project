%!TEX root = /Users/Nikolaj/Developer/GPU-Project/Report/Report.tex
%kig evt på en simpson integration når vi kigger på middle da det egentlig ikke er en differential ligning men et integrale
%simpson video: http://www.youtube.com/watch?v=ns3k-Lz7qWU

%kan det betale sig at køre dele af programmet på CPU'en?

%- Did we achieve what we wanted? what did we discover during the project? What can be changed in future implementations? \\

When computing reserves for several insurance holders one after another it could be beneficial to use tabulation and store the result of one or several models and the parameters they were given. Whenever a model is going to be solved with a specific input, the implementation could check if the result already exists and instantly return it instead of solving the model again. If the result was not found it could solve it and store the result for future use. \\ 

This approach is only beneficial when retrieving the result from storage is faster than solving the model. It is not beneficial if the amount of identical input parameters is too small. It would be interesting to examine the input parameter collision rate by solving the models for a large amount of insurance holders. The efficiency of this approach varies with the choice of data storage. \\

It could also be beneficial to have a mechanism for automatically determining the optimal amount of blocks and threads per block before running the kernel. An easy solution could be to implement an auto tuner that tests the running time with different amounts of blocks and threads per block and settles on the fastest. \\

If we examine Equation \ref{middlediff} used in the middle model it is a constant multiplied by an integral. This means that there is no need to use the Runge-Kutta method and we could have chosen another approach for integral approximating such as Simpson's rule \cite{simp}. Simpson's rule requires an even number of steps which our middle model will always have as long as the steps per year is an even integer. The Runge-Kutta method uses four calculations for each step ($k1$, $k2$, $k3$ and $k4$), whereas Simpson's rule only uses a single calculation. This could potentially make the estimation of the middle model four times faster but it also introduces the risk of reducing the precision in the result. In our implementation we have optimized the calculation of $k3$ as described in Section \ref{implementation} which means that Simpson's rule would only make the estimation of the middle model three times faster.
